\documentclass[11pt,reqno]{article}

\usepackage{mathtools, mathrsfs, mathdots}
\usepackage{amsmath, amsthm, amssymb, amsfonts, amscd}
\usepackage{stmaryrd}
\usepackage[mathscr]{euscript}
\usepackage{dsfont}
\usepackage{hyperref}

\usepackage{epsfig}

% \usepackage[a4paper,left=2.85cm,right=2.85cm,top=3cm,bottom=3cm,headsep=1.2cm]{geometry}

\usepackage{todonotes}

\usepackage{tikz} 
\usepackage{tikz-cd}
\usepackage{ytableau}
\usepackage{caption}

\usepackage[all]{xy}

\usepackage{latexsym}
\usepackage{palatino}

\usepackage{enumitem}

\usepackage[normalem]{ulem}
\usepackage{upgreek}
% \usepackage{mathpazo}

\usepackage{cleveref}

\usepackage{graphicx,subfigure}

\usepackage{setspace}

% \spacing{1.05}

\linespread{1.2}

\oddsidemargin=0in
\evensidemargin=0in
\textwidth=6.50in             % paper is 8.50in wide

% 1in margins at top and bottom
\headheight=10pt
\headsep=10pt
\topmargin=.25in
\textheight=8in


%%%%%%%%%%%%%%%%%%%%%%%%%%%%%%%%%%%%%%%%%%%%%%%%%%%%%%%%%%%%%%%%%%%%%%%%%%
%%%%%%%%%%%%%%%%%%%%%%%%%%%%%%%%%%%%%%%%%%%%%%%%%%%%%%%%%%%%%%%%%%%%%%%%%%
%%%%%%%%%%%%%%%%%%%%%%%%%%%%%%%%%%%%%%%%%%%%%%%%%%%%%%%%%%%%%%%%%%%%%%%%%%


\newtheorem{thm}{Theorem}[section]
\newtheorem{prop}[thm]{Proposition}
\newtheorem{lem}[thm]{Lemma}
\newtheorem{cor}[thm]{Corollary}
\newtheorem{conj}[thm]{Conjecture}
\newtheorem{prob}[thm]{Problem}
\newtheorem{ques}[thm]{Question}
\theoremstyle{definition}
\newtheorem{defn}[thm]{Definition}
\newtheorem{exam}[thm]{Example}
%% \theoremstyle{remark}
\newtheorem{rmk}[thm]{Remark}
\newtheorem{obsv}[thm]{Observation}
\theoremstyle{remark}
\newtheorem*{notation}{Notations}

\numberwithin{equation}{section}
\numberwithin{thm}{subsection}

%Blackboard bold letters.
\renewcommand{\AA}{\mathbb{A}}
\newcommand{\BB}{\mathbb{B}}
\newcommand{\CC}{\mathbb{C}}
\newcommand{\DD}{\mathbb{D}}
\newcommand{\EE}{\mathbb{E}}
\newcommand{\FF}{\mathbb{F}}
\newcommand{\GG}{\mathbb{G}}
\newcommand{\HH}{\mathbb{H}}
\newcommand{\II}{\mathbb{I}}
\newcommand{\JJ}{\mathbb{J}}
\newcommand{\KK}{\mathbb{K}}
\newcommand{\LL}{\mathbb{L}}
\newcommand{\MM}{\mathbb{M}}
\newcommand{\NN}{\mathbb{N}}
\newcommand{\OO}{\mathbb{O}}
\newcommand{\PP}{\mathbb{P}}
\newcommand{\QQ}{\mathbb{Q}}
\newcommand{\RR}{\mathbb{R}}
\renewcommand{\SS}{\mathbb{S}}
\newcommand{\TT}{\mathbb{T}}
\newcommand{\UU}{\mathbb{U}}
\newcommand{\VV}{\mathbb{V}}
\newcommand{\WW}{\mathbb{W}}
\newcommand{\XX}{\mathbb{X}}
\newcommand{\YY}{\mathbb{Y}}
\newcommand{\ZZ}{\mathbb{Z}}

%Calligraphic letters
\newcommand{\cA}{\mathcal{A}}
\newcommand{\cB}{\mathcal{B}}
\newcommand{\cC}{\mathcal{C}}
\newcommand{\cD}{\mathcal{D}}
\newcommand{\cE}{\mathcal{E}}
\newcommand{\cF}{\mathcal{F}}
\newcommand{\cG}{\mathcal{G}}
\newcommand{\cH}{\mathcal{H}}
\newcommand{\cI}{\mathcal{I}}
\newcommand{\cJ}{\mathcal{J}}
\newcommand{\cK}{\mathcal{K}}
\newcommand{\cL}{\mathcal{L}}
\newcommand{\cM}{\mathcal{M}}
\newcommand{\cN}{\mathcal{N}}
\newcommand{\cO}{\mathcal{O}}
\newcommand{\cP}{\mathcal{P}}
\newcommand{\cQ}{\mathcal{Q}}
\newcommand{\cR}{\mathcal{R}}
\newcommand{\cS}{\mathcal{S}}
\newcommand{\cT}{\mathcal{T}}
\newcommand{\cU}{\mathcal{U}}
\newcommand{\cV}{\mathcal{V}}
\newcommand{\cW}{\mathcal{W}}
\newcommand{\cX}{\mathcal{X}}
\newcommand{\cY}{\mathcal{Y}}
\newcommand{\cZ}{\mathcal{Z}}

%Fraktur letters
\newcommand{\fb}{\mathfrak{b}}
\newcommand{\fg}{\mathfrak{g}}
\newcommand{\fh}{\mathfrak{h}}
\newcommand{\fm}{\mathfrak{m}}
\newcommand{\fn}{\mathfrak{n}}
\newcommand{\ft}{\mathfrak{t}}
\newcommand{\fP}{\mathfrak{P}}
\newcommand{\fQ}{\mathfrak{Q}}
\newcommand{\fS}{\mathfrak{S}}

%tilde letters
\newcommand{\tS}{\widetilde{S}}
\newcommand{\ts}{\widetilde{s}}
\newcommand{\tA}{\widetilde{A}}
\newcommand{\tG}{\widetilde{G}}
\newcommand{\tR}{\widetilde{R}}
\newcommand{\tPio}{\widetilde{\Pio}}

%bold letters
\newcommand\xx{\mathbf{x}}
\newcommand\yy{\mathbf{y}}
\newcommand\uu{\mathbf{u}}
\newcommand\mm{\mathbf{m}}

%sf letters
\newcommand\ww{\mathsf{w}}
\newcommand\vv{\mathsf{v}}
\newcommand\zz{\mathsf{z}}



\newcommand\ch{\operatorname{ch}}
\newcommand\tr{\operatorname{tr}}
\newcommand\GL{\operatorname{GL}}

\newcommand\equivclass[1]{#1/{\sim}}
\newcommand\qand{\quad\mbox{and}\quad}
\newcommand\mand{\mbox{ \& }}

\newcommand\CHK[1]{\textcolor{red}{#1}}

\title{Dummit--Foote Exercises}
\date{\today}

\spacing{1.05}

\begin{document}

\maketitle

\section{Chapter 1: Introduction to Groups}

\subsection{Basic Axioms and Examples}

\newpage

\section{Chapter 2: Subgroups}

\newpage

\section{Chapter 3: Quotient Groups and Homomorphisms}

\newpage

\section{Chapter 4: Group Actions}

\newpage

\section{Chapter 5: Direct and Semidirect Products and Abelian Groups}

\newpage

\section{Chapter 6: Further Topics in Group Theory}

\newpage

\section{Chapter 7: Introduction to Rings}

\newpage

\section{Chapter 8: Euclidean Domains, Principal Ideal Domains, and Unique Factorization Domains}

\newpage

\section{Chapter 9: Polynomial Rings}

\newpage

\section{Chapter 10: Introduction to Module Theory}

\newpage

\section{Chapter 11: Vector Spaces}

\newpage
\section{Chapter 12: Module Theory over Principal Ideal Domains}

\newpage

\section{Chapter 13: Field Theory}

\newpage

\section{Chapter 14; Galois Theory}

\newpage


\section{Chapter 15: Commutative Rings and Algebraic Geometry}

\newpage

\section{Chapter 16: Artinian Rings, Discrete Valuation Rings, and Dedekind Domains}


\newpage

\section{Chapter 17: Introduction to Homological Algebra and Group Cohomology}

\newpage

\section{Chapter 18: Representation Theory and Character Theory}

\subsection{Linear Actions and Modules over Group Rings}
\subsection{Wedderburn's Theorem and Some Consequeces}
\subsection{Character Theory and the Orthogonality Relations}

\begin{prob}
    Prove that \( \tr AB = \tr BA \) for \( n \times n \) matrices \( A \) and \( B \) with entries 
    from any commutative ring.
\end{prob}
  
\input{DF_18/DF_18_3_2.tex}
\input{DF_18/DF_18_3_3.tex}
\input{DF_18/DF_18_3_4.tex}
\input{DF_18/DF_18_3_5.tex}
\begin{prob}
    Let \( G \) be a finite group.
    Let \( \phi: G\rightarrow \GL(V) \) be a representation with character \( \psi \) over \( \CC \).
    Let \( W \) be the subspace \( \{v\in V \mid \mbox{\( \phi(g)(v) = v  \) for all \( g\in G \)}\} \)
    of \( V \) fixed pointwise by all elements of \( G \). Prove that \( \dim W = (\psi, \chi_1) \),
    where \( \chi_1 \) is the principal character of \( G \).
\end{prob}

\newpage

\section{Chapter 19: Examples and Applications of Character Theory}

\end{document}
